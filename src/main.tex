\documentclass[a4paper, UTF8]{article}
\usepackage{ctex}
\usepackage{geometry} %改变页面格式
\usepackage{CJKnumb} %中文计数
\usepackage{titlesec} %section标题格式设置
\usepackage{setspace}
\usepackage{titling}
%---------GLOBAL Settings----------------
\pagestyle{plain} %取消页眉、编号
\titleformat{\section}{\Fakeheiti\zihao{4}}{\CJKnumber{\thesection}、}{0em}{}
\ctexset{
	autoindent = true % 即便字体大小变化,首段自动缩进相应两个字体字宽的长度
}
%-------------MACRO-----------------------------
%字体设置 (see document of xeCJK for more informations)
\newCJKfontfamily\Fakesongti{宋体}[AutoFakeBold] %宋体 = SimSun, \Fakesongti = 可以加粗的“伪宋体”
\newCJKfontfamily\Fakeheiti{黑体}[AutoFakeBold]

%Print a specific table with 3 items
\newcommand{\MyAuthor}[7]{
	\centerline{
        \Fakesongti \zihao{5}
        #1 \quad
        #2 \quad #3 \quad
        #4 \quad #5 \quad
        #6 \quad #7
    }
}

\begin{document}

\newgeometry{top=2.54cm, bottom=2.54cm, left=3.17cm, right=3.17cm}

%------------封面------------------------
	\clearpage\thispagestyle{empty}\addtocounter{page}{-1}

    \vspace*{2.2cm}

    \centerline{\Fakesongti \zihao{1} \bfseries 2019第十二届“深大杯”数学建模竞赛}

    \vspace*{3.05cm}

    \MyAuthor{某某学院}{姓名}{某某}{学号}{0000000000}{电话}{11100000000}
    
    \MyAuthor{某某学院}{姓名}{某某}{学号}{0000000000}{电话}{11100000000}
    
    \MyAuthor{某某学院}{姓名}{某某}{学号}{0000000000}{电话}{11100000000}
	
	\clearpage
%------------正文部分------------------------
	
	\newpage
    
    \phantom{\Fakesongti \zihao{4} 无}

    \centerline{\Fakeheiti \zihao{3} 标题}

    \phantom{\Fakesongti \zihao{5} 无}

    \centerline{\Fakeheiti \zihao{4} \bfseries 摘要}

    \Fakesongti \zihao{-4}

    将你们在文章中的主要内容作一个简单陈述。字数大概在200~300字之间,
    要求突出重点,说出你们解决问题的途径和简单的结论。
    
    \phantom{无 \\ 无}
    
    {\noindent 关键词:无无无\phantom{无无}无无无无}

    \phantom{无 \\ 无}

    \Fakesongti \zihao{-4}

	\section{问题的提出}

        将问题用你们的理解重新进行表述
    
    \section{问题的分析}
    
        对问题进行分析,包括你们查阅的文献中对此问题或近似问题的解决方法的综述,
        还可能包括定性分析和定量分析,并讨论对此问题你们拟采用的方法。
	
    \section{模型假设}

        建立数学模型解决问题都是有一定条件的,对此,经过你们的分析,作出你们针对此问题的合理的假设。
	
	\section{模型的建立与求解}
    
        在这里写出你们对此问题建立数学模型并求解的过程,需要分成若干个校标题,
        将你们建模的过程和求解的过程详细但不啰嗦地说清楚,如果用的是已经存在的某个数学模型,
        则需要将模型的基本知识做个简单介绍。

    \section{模型评价}
    
        针对你们对此问题建立模型并计算得到的相关结论进行讨论,讨论你们得到的数值是否合理,
        模型是否合理,在什么地方可能改进等等。
	
\end{document}
